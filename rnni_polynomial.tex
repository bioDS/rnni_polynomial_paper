\documentclass{amsart}

% \usepackage[notref,notcite]{showkeys}
\usepackage[style=authoryear,ibidtracker=false,uniquename=false,giveninits=true,terseinits=true,maxbibnames=5,backend=biber]{biblatex}
\usepackage{float}
\usepackage{graphicx}
\usepackage{todonotes}
\usepackage{subcaption}


\renewbibmacro{in:}{}
\addbibresource{rnni_polynomial.bib}

\newtheorem{lemma}{Lemma}
\newtheorem{theorem}{Theorem}

\newcommand{\rnni}{\mathrm{RNNI}}
\newcommand{\findpath}{\textsc{FindPath}}
\newcommand{\mrca}{\mathrm{mrca}}
\newcommand{\rank}{\mathrm{rank}}
\newcommand{\nni}{\mathrm{NNI}}

\graphicspath{{figures/}}

\begin{document}


\begin{lemma}
    Let $T$ and $R$ be ranked trees.
    If for all neighbours $T' \in N_1(T)$ of $T$ it is $d_{FP}(T',R) \geq d_{FP}(T,R) - 1$, then it is $d_{\rnni}(T,R) = d_{FP}(T,R)$.
\end{lemma}

\proof
    We prove this lemma by induction on the distance $d_{\rnni}(T,R)$.
    It is easy to see that the statement is true for the base case $d_{\rnni}(T,R) = 1$.
    Let us consider trees $T$ and $R$ with distance $d_{\rnni}(T,R) = d+1$ and assume that the lemma is true for all pairs of trees with distance less than $d+1$.
    For contradiction we assume that $d_{FP}(T,R) > d_{\rnni}(T,R) = d+1$.
    There must be a tree $T'' \in N_1(T)$ that has $\rnni$ distance $d_{\rnni}(T'',R) = d$.
    With the induction hypothesis it follows $d_{FP}(T'',R) = d$, which contradicts that it is $d_{FP}(T',R) \geq d_{FP}(T,R) - 1 > d$ for all trees $T' \in N_1(T)$.
    Therefore, it cannot be $d_{FP}(T,R) > d_{\rnni}(T,R)$ which concludes the proof of the lemma.
\endproof


\begin{theorem}
Let $T,R$ be trees and $p = \findpath(T,R)$ the path from $T$ to $R$ computed by $\findpath$.
Let $d_{FP}(T,R)$ denote the $\findpath$
\todo{better not call it distance}
distance from $T$ to $R$, that is the length of $p$.
For all $T' \in N_1(T)$ it is $d_{FP}(T',R) \geq d_{FP}(T,R) - 1$, where $N_1(T)$ is the one-neighbourhood of $T$.
\end{theorem}

\proof
For contradiction we assume that for the pair $T,R$ of trees there is a tree $T' \in T$ with $d_{FP}(T',R) < d_{FP}(T,R) - 1$.
We also assume that $T,R$ is the pair with minimum $\findpath$ distance $d_{FP}(T,R)$ among those trees fulfilling this property.
It immediately follows that the first move on $p = \findpath(T,R)$ changes one of the nodes $v,w$ that bound the interval $[v,w]$ on which the $\rnni$ move between $T$ and $T'$ is performed.
\todo{introduce notion of interval}
This is due to the fact that otherwise the first steps on $p$ and $p' = \findpath(T',R)$ coincide, as $T$ and $T'$ only differ by the interval $[v,w]$, which contradicts the assumption that $T$ and $R$ are the closest pair of trees for which we can find such a $T'$.

In this proof we will at first distinguish the case that there is an $\nni$ move between $T$ and $T'$ from the case that there is a rank swap.
For each of these cases we further distinguish between all moves possible on the tree $T$ on $p$, which happen on the intervals adjacent to $[v,w]$.
Note that $p$ is the path computed by $\findpath$, so there is a cluster $C_k$ whose most recent common ancestor moves down by the $\rnni$ move following $T$ on $p$.
We denote the tree following $T$ on $p$ by $\hat T$.

At first we consider the case that there is an $\nni$ move between $T'$ and $T$ as illustrated in the top of Figure~\ref{fig:thm_fp_nni1}.
It follows that $(v,w)$ is an edge in $T$.
Let us now distinguish different types of moves between $T$ and $\hat T$ on intervals incident to $v$ or $w$.

\begin{enumerate}
    \item $\nni$ move on edge $(v,w)$

    If this $\nni$ move results in $\hat T = T'$, it is $d_{FP}(T',R) = d_{FP}(T,R) - 1$, which contradicts the assumption on the choice of $T'$.
    Otherwise $\hat T$ follows $T'$ on $p'$ as well, meaning that the remaining part of $p$ and $p'$ coincide, and it follows $d_{FP}(T',R) = d_{FP}(T,R)$.
    This is true as $(C_k)_T$ that is moved down by $\findpath$ on this move must contain taxa in one of the subtrees below $v$ and in the subtree below $w$ that does not contain $v$, for example $B$ and $C$ in Figure~\ref{fig:thm_fp_nni1}.
    And as $(C_k)_T'$ moves down on $p'$ as well, the move made on $T'$ is an $\nni$ move that results in the same tree as the one following $T$ on $p$.
    This is again a contradiction to the assumption $d_{FP}(T',R) < d_{FP}(T,R) - 1$.

    \begin{figure}[!hbt]
    \centering
    \includegraphics[width=0.4\textwidth]{thm_fp_nni1}
    \vspace{12pt}
    \caption{$\nni$ move between $T$ and $T'$ on the dashed edge $(v,w)$ and the third $\rnni$ neighbour resulting from a move on that edge.}
    \label{fig:thm_fp_nni1}
    \end{figure}

    \item $\nni$ moves on edge $(u,v)$ above $(v,w)$

    Notice that this is only possible if the interval above $(v,w)$ is an edge.
    As it is depicted in the top of Figure~\ref{fig:thm_fp_nni2a}, there are two $\nni$ moves on $(u,v)$ possible that lead to different trees.
    We denote the subtrees that are children of $w$ by $A$ and $B$, the one of $v$ not containing $w$ by $C$ and the one of $u$ not containing $v$ by $D$, according to the labels in Figure~\ref{fig:thm_fp_nni2a}.
    \todo{We call both cluster and subtrees $A,B,C$ and $D$}

    \begin{figure}[H]
        \begin{subfigure}[b]{.45\textwidth}
            \centering
            \includegraphics[width=0.9\linewidth]{thm_fp_nni2a.eps}
            \vspace{12pt}
            \caption{path $p$ following $T$}
            \label{fig:thm_fp_nni2a}
        \end{subfigure}
        \begin{subfigure}[b]{.45\textwidth}
            \centering
            \includegraphics[width=0.9\linewidth]{thm_fp_nni2b.eps}
            \vspace{12pt}
            \caption{path $p'$ following $T'$}
            \label{fig:thm_fp_nni2b}
        \end{subfigure}
        \caption{Comparison of $p$ and $p'$ if there is an $\nni$ move on the edge $f$ above $e$ in $T$.
        The trees on the bottom are those following $T$ and $T'$ on $p$ and $p'$ after two $\rnni$ moves, respectively, depending on the cluster $C_k$ that is currently considered on $\findpath$:
        ${C_k \subseteq A \cup D}$ on the left, ${C_k \subseteq B \cup D}$ in the middle, ${C_k \subseteq C \cup D}$ on the right.}
    \end{figure}

    All moves on edge $(u,v)$ of $T$ that could happen on $p$ are depicted in Figure~\ref{fig:thm_fp_nni2a}.
    We will now consider each of these moves separately.
    If $\hat T$ is the tree at the top left of Figure~\ref{fig:thm_fp_nni2a}, which results from $T$ by exchanging the subtrees $C$ and $D$, the cluster $C_k$ that is moved down by $\findpath$ is either a subset of $A \cup B \cup D$, $A \cup D$ or $B \cup D$.

    \begin{enumerate}
        \item
            If $C_k \subseteq A \cup B \cup D$, then $C_k$ reached it's final position in $\hat T$ on $p$.
            Specifically, the subtree containing $A \cup B$ in $T$ is the same subtree in $R$, by the nature of $\findpath$.
            It follows that on $p'$ there is a move that moves the most recent common ancestor of $A \cup B$ down right before $C_k$ is considered.
            This means that the tree following $T'$ on $p'$ is $T$, contradicting $d_{FP}(T',R) < d_{FP}(T,R) - 1$.
        \item
            If $C_k \subseteq A \cup D$, the move on $\hat T$ on $p$ moves $(C_k)_{\hat T}$ further down by exchanging $B$ and $D$, resulting in the tree in the bottom left of Figure~\ref{fig:thm_fp_nni2a}.
            As the same most recent common ancestor $(C_k)_{T'} \subseteq A \cup D$ moves down on $p'$, the two moves following $T'$ move the subtree $D$ down by exchanging it with its neighbours $B$ and $C$ as depicted on the left of Figure~\ref{fig:thm_fp_nni2b}.
            Comparing the trees on $p$ and $p'$ after the two moves following $T$ and $T'$, respectively, shows that we are again at a stage where two trees on these paths only differ by one edge (dotted edges in the trees at the bottom left of Figures~\ref{fig:thm_fp_nni2a} and~\ref{fig:thm_fp_nni2b}).
            Furthermore, the relation of the distancesof these tree to $R$ are the same as the ones from $T$ and $T'$ to $R$, which contradicts the assumption that $T$ and $R$ are the closest pair of trees with this relation.
        \item
            If $C_k \subseteq B \cup D$, the two $\rnni$ moves following $T$ and $T'$ on $p$ and $p'$, respectively, end up in the trees depicted in the tree in the middle at the bottom of Figures~\ref{fig:thm_fp_nni2a} and \ref{fig:thm_fp_nni2b}, analogous to the previous case.
            As in the previous case, these trees coincide in all but one interval (dotted edges in the figure), which again contradicts the assumptions on $T$ and $R$.
    \end{enumerate}

    If the $\nni$ move on $(u,v)$ results in a tree $\hat T$ containing a subtree $C \cup D$ as illustrated on the top right of Figure~\ref{fig:thm_fp_nni2a}, it is $C_k \subseteq C \cup D$.
    If $(C_k)_T$ does not move further down on $p$, it follows that $A \cup B$ is a cluster in $R$ and that before $C_k$ is considered on $p'$, $(A \cup B)_{T'}$ moves down by one $\rnni$ move.
    This means that $T$ follows $T'$ on $p'$, which contradicts $d_{FP}(T',R) < d_{FP}(T,R) - 1$.
    If on the other side the rank of $(C_k)_T$ decreases further on $p$, the move on $\hat T$ is a rank swap as depicted in the bottom right of Figure~\ref{fig:thm_fp_nni2a}.
    The moves on $p'$ that decreases the rank of $(C_k)_{T'}$ are $\nni$ moves exchanging $D$ with $B$ and $A$, because it is $C_k \subseteq C \cup D$.
    These moves are shown on the right of Figure~\ref{fig:thm_fp_nni2b}.
    As above, the two trees resulting from the two moves following $T$ and $T'$ on $p$ and $p'$, respectively, coincide by all but one interval.
    Therefore, we end up in the same contradiction as above.

    \item Rank move on interval $[u,v]$ above $(v,w)$

    Notice that this is only possible if $[u,v]$ is a rank interval.
    If after the rank move, which increases the rank of $v$, there is a rank move on $\hat T$ that increases the rank of $w$, the same rank moves happen on $p'$ and the trees after these two moves on $p$ and $p'$ coincide in all but one interval.
    As previously, this contradicts the assumption that $T$ and $R$ are the closest trees where there is a $T' in N_1(T)$ with $d_{FP}(T',R) < d_{FP}(T,R) - 1$.
    If on the other side there is no such rank move on $\hat T$, then the cluster induced by $w$ is a cluster in $R$ as well.
    The tree following $T'$ on $p'$ is $T$ as it is the result of building this cluster in $T'$, which happens right before the cluster $C_k$ is being considered on $p'$.
    This is due to the order in which $\findpath$ considers most recent common ancestors of clusters.
    This contradicts $d_{FP}(T',R) < d_{FP}(T,R) - 1$.

    \item $\rnni$ moves on interval below $[v,w]$

    If there is a move on the interval below $[v,w]$, $C_k$ is a subset of $A \cup B$, following the notions of the trees in the top of Figure~\ref{fig:thm_fp_nni1}.
    In this case, the move following $T'$ on $p'$ exchanges $B$ and $C$ by an $\nni$ move first, which moves $(C_k)_{T'}$ down and transforms $T'$ into $T$, contradicting $d_{FP}(T',R) < d_{FP}(T,R) - 1$.
\end{enumerate}

Let us now assume that there is a rank move between $T$ and $T'$ where nodes $v$ and $w$ inducing clusters $A$ and $B$ swap ranks as depicted at the top of Figure~\ref{fig:thm_fp_rank1}.
We will again consider the different moves from $T$ to $\hat T$ that are possible on intervals adjacent to $(v,w)$, which is the interval of the rank move between $T$ and $T'$.


\begin{figure}[!hbt]
\centering
\includegraphics[width=0.4\textwidth]{thm_fp_rank1}
\vspace{12pt}
\caption{Rank move between $T$ and $T'$ on the interval given by the highlighted nodes, and an $\nni$ move on $T$ if the edge $(u,v)$ has length one.}
\label{fig:thm_fp_rank1}
\end{figure}

\begin{enumerate}
    \item Rank move on $[v,w]$

    The only possible move on $[v,w]$ on $T$ is a rank move resulting in $\hat T = T'$.
    It follows $d_{FP}(T',R) = d_{FP}(T,R) - 1$, which obviously contradicts our assumption $d_{FP}(T',R) < d_{FP}(T,R) - 1$.

    \item $\nni$ move on edge above $[v,w]$

    Notice that this is only possible if the interval above $[v,w]$ is an edge $(u,v)$.
    We are now going to distinguish the case that $u$ is the parent of $w$ from the case that it is not.
    At first we assume that $u$ is parent of $w$.
    It follows, that after the $\nni$ move on $(u,v)$, there is a new cluster containing $A$ and $B_1$, which is one of the subtrees that are children of $v$.
    This move is depicted on the left of Figure~\ref{fig:thm_fp_rank1}.
    If such a move happens the currently considered cluster $C_k$ must be subset of $A \cup B_1$, which means that $A$ is already at the same place in $T$ as in $R$.
    Therefore, the move on $p'$ on $T'$ must be a rank swap leading to $T$, as $(A)_{T'}$ is moved to its correct position before $C_k$ is considered, which follows from how $\findpath$ works.
    However, this means that $T$ is on the path $p'$ form $T'$ to $R$, which is a contradiction to $d_{FP}(T',R) < d_{FP}(T,R) - 1$

    Let us now assume that $u$ is not the parent of $w$.
    In this case the $\nni$ move builds a new cluster containing taxa of $B$ and of the subtree $C$ that is the child of $u$ that does not contain $v$, as depicted in the top left of Figure~\ref{fig:thm_fp_rank2}.
    It follows that $C_k \subseteq C \cup B_1$ where $B_1$ is one of the subtrees that is child of $w$.
    If there was a rank move on $\hat T$ decreasing the rank of $(C_k)_{\hat T}$, the node inducing cluster $A$ has the same rank in $T$ as in $R$.
    In this case the tree following $T'$ on $p'$ must be $T$ as $(A)_{T'}$ moves to it's final position on $\findpath$ before $C_k$ is moved down.
    However, this contradicts $d_{FP}(T',R) < d_{FP}(T,R) - 1$.
    Let us now consider what happens if there is a rank move on $\hat T$ that decreases the rank of $(C_k)_{\hat T}$ on $p$.
    This case is depicted on the left of Figure~\ref{fig:thm_fp_rank2}.
    The moves happening on $p'$ are as follows.
    First the rank of $(C_k)_{T'}$ decreases by a rank swap of $(C \cup B)_{T'}$ and $(A)_{T'}$ on $T'$ and then an $\nni$ move exchanges $B_2$ and $C$, as it is depicted on the right of the same figure.
    One can easily see that the two trees on $p$ and $p'$ that are two $\rnni$ moves apart from $T$ and $T'$, respectively, only differ by one interval.
    Again, this is a contradiction to the fact that $T$ and $R$ are the trees with minimum distance for which we can find a $T' \in N_1(T)$ with $d_{FP}(T',R) < d_{FP}(T,R) - 1$.

    \begin{figure}[!hbt]
    \centering
    \includegraphics[width=0.4\textwidth]{thm_fp_rank2}
    \vspace{12pt}
    \caption{Comparison of $p$ (left) and $p'$ (right) if there is a rank move between $T$ and $T'$ and an $\nni$ move on the edge below the corresponding rank interval follows on $p$.}
    \label{fig:thm_fp_rank2}
    \end{figure}

    \item Rank move on interval above $[v,w]$

    Notice that this is only possible if the interval above $[v,w]$ is a rank interval.
    If there is a rank move increasing the rank of $v$, and no rank move increasing the rank of $w$ immediately afterwards, $C_k$ reaches it's final position in the tree $\hat T$ on $p$ following $T$.
    It follows that the cluster $A$ induced by $w$ in $T$ is a cluster in $R$ as well.
    This means that the move following $T'$ on $p'$ leads to $T$, as the node inducing $A$ moves down before $C_k$ is considered on $\findpath$.
    This contradicts $d_{FP}(T',R) < d_{FP}(T,R) - 1$.
    If on the other side the rank swap on $T$ is directly followed by a rank swap increasing the rank of $w$, the same moves happen on $p'$ and the trees following $T$ and $T'$ after two $\rnni$ moves on $p$ and $p'$, respectively, coincide in all but one interval.
    And this again is a contradiction to our choice of $T$ and $R$.

    \item $\rnni$ moves on interval below $[v,w]$

    If there is a move on the interval below $[v,w]$, it follows that $\findpath$ moves $C_k \subseteq A$ down where $A$ is the cluster induced by $w$.
    For decreasing the rank of the most recent common ancestor of $C_k \subseteq A$ in $T'$, the move on $T'$ must be a rank swap that results in $T$, contradicting  $d_{FP}(T',R) < d_{FP}(T,R) - 1$.
\end{enumerate}

We can conclude that in any of the above cases we end up in a contradiction, proving that a tree $T'$ with $d_{FP}(T',R) < d_{FP}(T,R) - 1$ does not exist in $N_1(T)$.
Therefore we can follow that $d_{FP}(T',R) \geq d_{FP}(T,R) -1$ for all $T' \in N_1(T)$.
\endproof

\end{document}
