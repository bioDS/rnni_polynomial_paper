\documentclass{amsart}

\usepackage[notref,notcite]{showkeys}
\usepackage[style=authoryear,ibidtracker=false,uniquename=false,giveninits=true,terseinits=true,maxbibnames=5,backend=biber]{biblatex}
\usepackage{float}
\usepackage{graphicx}
\usepackage{todonotes}
\usepackage{subcaption}
\usepackage{amsmath}
\usepackage{amsthm}
\usepackage{amssymb}
\usepackage[foot]{amsaddr}
\usepackage[misc]{ifsym}
\usepackage{enumerate}

\renewbibmacro{in:}{}
\addbibresource{rnni_polynomial.bib}

\newtheorem{lemma}{Lemma}
\newtheorem{theorem}{Theorem}

\newcommand{\rnni}{\mathrm{RNNI}}
\newcommand{\findpath}{\textsc{FindPath}}
\newcommand{\mrca}{\mathrm{mrca}}
\newcommand{\rank}{\mathrm{rank}}
\newcommand{\nni}{\mathrm{NNI}}
\newcommand{\fp}{\mathrm{FP}}
\renewcommand{\O}{\mathcal O}

\graphicspath{{figures/}}


\title[Computing $\rnni$ distance]{Efficient algorithm for computing nearest neighbour interchange distance between ranked phylogenetic trees}
\date{\today}
\author{Lena Collienne\textsuperscript{1}}
\email{lena.collienne@postgrad.otago.ac.nz}
\address{\textsuperscript{1}Department of Computer Science, University of Otago, New Zealand}
\author{Alex Gavryushkin\textsuperscript{1, \Letter}}
\email{\textsuperscript{\Letter}alex@biods.org}


\begin{document}

\begin{abstract}
We present a quadratic algorithm to compute shortest paths, and hence the distance, between ranked phylogenetic trees under the ranked nearest neighbour interchange operation.
\end{abstract}


\maketitle

Things to introduce:

\begin{itemize}
\item Trees with the notion of an interval in a tree (if we want intervals), clusters, clusters induced by nodes, subtrees induced by clusters, list representation of trees.
\item $(T)_k$: node of rank k, $(C)_T$: mrca of cluster $C$ in $T$
\item $\findpath$, including its time complexity.
\item Note that $\findpath$ is deterministic, that is, an interval and a move on the interval are uniquely determined (this is necessary to concentrate on the $v, w$ nodes in the proof of main theorem).
\end{itemize}

\begin{theorem}
The time complexity of computing the $\rnni$ distance between trees on $n$ leaves is $\O(n^2)$.
\end{theorem}

\proof
We prove this theorem by showing that for every pair of trees $T$ and $R$, the path computed by the $\findpath$ algorithm is a shortest $\rnni$ path.
We denote this path by $\fp(T, R)$ and its length by $|\fp(T, R)|$.

Assume to the contrary that $T$ and $R$ are two trees with a minimum possible distance $d(T, R)$ such that $d(T,R) \neq |\fp(T,R)|$, that is, $d(T,R) < |\fp(T,R)|$.
Let $T'$ be the first tree on a shortest $\rnni$ path from $T$ to $R$.
Then $d(T',R) = d(T, R) - 1$ and the distance between $T'$ and $R$ is strictly smaller than that between $T$ and $R$.
Hence $d(T', R) = |\fp(T',R)| < |\fp(T,R)| - 1$.
We finish the proof by showing that no trees satisfy this inequality.

Specifically, we will show that for all trees $T$, $R$, and $T'$ such that $T'$ is one $\rnni$ move away from $T$,
\begin{equation}
	|\fp(T',R)| \geq |\fp(T,R)| - 1
 	\label{eqn:iff_inequality}
\end{equation}

We will use Figure~\ref{fig:proof_idea} to demonstrate our argument.

\begin{figure}[!hbt]
\centering
\includegraphics[width=0.6\textwidth]{proof_idea}
\vspace{12pt}
\caption{Trees $T$, $T'$, and $R$ contradicting inequality~(\ref{eqn:iff_inequality}).
$\fp(T,R) = [T,T_1,T_2, \ldots, R]$ and $\fp(T',R) = [T',T'_1,T'_2, \ldots, R]$}
\label{fig:proof_idea}
\end{figure}

Assume to the contrary that $T$ and $R$ are trees for which there exists $T'$ violating inequality~(\ref{eqn:iff_inequality}).
Out of all such pairs $T, R$ choose one with the minimal $|\fp(T, R)|$.
Denote $\fp(T,R)$ by $p$ and $\fp(T', R)$ by $p'$, and let $[(T)_t, (T)_{t+1}]$ be the interval in $T$ on which the $\rnni$ move connecting $T$ and $T'$ is performed.
As every node of rank $s \notin \{t, t+1\}$ induces the same cluster in $T$ and $T'$, $\findpath$ would make the exact same rearrangement in both trees if any of those nodes is involved in the first move required by the algorithm.
Let $C_k$ denote the cluster of $R$ such that the node $(C_k)_T$ is moved down by the first move on $p$.
If the rank of $(C_k)_T$ is not in $\{t, t+1\}$, the first moves on both $p$ and $p'$ result in the same rearrangement of clusters and give rise to trees which we denote by $T_1$ and $T_1'$, respectively, as in Figure~\ref{fig:proof_idea}.
Since in this case $T_1$ and $T_1'$ are $\rnni$ neighbours, $\fp(T_1, R)$  and $\fp(T_1', R)$ violate inequality~(\ref{eqn:iff_inequality}) but $\fp(T_1, R)$ is strictly shorter than $\fp(T, R)$, contradicting our minimality assumption.
Hence, the first move on $p$ has to involve an interval incident to at least one of the nodes $(T)_t$, $(T)_{t+1}$.

We will distinguish two cases depending on whether $T$ and $T'$ are connected by an $\nni$ or a rank move.
For each of these we will further distinguish all possible moves between $T$ and $T_1$.

\textbf{Case 1.} Let $T$ and $T'$ be connected by an $\nni$ move as illustrated in Figure~\ref{fig:thm_fp_nni1}.
It follows that $((T)_{t+1},(T)_t)$ is an edge in $T$.
We denote the clusters induced by the children of $(T)_t$ by $A$ and $B$ and the cluster that is induced by the child of $(T)_{t+1}$ that is not $(T)_t$ by $C$, as it is depicted in Figure~\ref{fig:thm_fp_nni1}.
We assume without loss of generality that the $\nni$ move from $T$ to $T'$ exchanges the subtrees induced by clusters $A$ and $B$.
In the following we distinguish all moves on $T$ that are possible on $p$, that are moves on intervals incident to $(T)_t$ or $(T)_{t+1}$, and show that either of them ends in a contradiction, which proves that inequality~(\ref{eqn:iff_inequality}) cannot be violated if $T$ and $T'$ are $\nni$ neighbours.

\begin{enumerate}[{1.}1.]
\item $\nni$ move on $((T)_{t+1},(T)_t)$

If this $\nni$ move results in $T_1 = T'$, it is $|\fp(T',R)| = |\fp(T,R)| - 1$, which contradicts our choice of $T'$.
If otherwise $T_1 \neq T'$, the cluster $B \cup C$ is built in $T_1$, as depicted in the bottom of Figure~\ref{fig:thm_fp_nni1}.
It follows that the cluster $C_k$ that is currently considered by $\findpath$ contains taxa in $B$ and $C$.
As the same cluster $C_k$ is considered in $\findpath$ on $T'$
\todo{LC: Do we need to explain this?}
 and the move on $T'$ decreasing the rank of $(C_k)_{T'}$ results in the same tree $T_1$, it is $T'_1 = T_1$.
Hence $|\fp(T',R)| = |\fp(T,R)|$, which contradicts $|\fp(T',R)| < |\fp(T,R)| - 1$.

\begin{figure}[!hbt]
\centering
\includegraphics[width=0.4\textwidth]{thm_fp_nni1}
\vspace{12pt}
\caption{$\nni$ move between $T$ and $T'$ on the dashed edge $((T)_{t+1},(T)_t)$ and the third $\rnni$ neighbour resulting from a move on that edge.}
\label{fig:thm_fp_nni1}
\end{figure}

\item $\nni$ move on edge $((T)_{t+2},T_{t+1})$

Notice that this is only possible if the interval $[(T)_{t+2},T_{t+1}]$ is an edge, we will consider the case that this is a rank interval later in this proof.
We stick to our notions of clusters $A,B,$ and $C$ as introduced above and additionally denote the cluster induced by the child of $(T)_{t+2}$ that is not $(T)_{t+1}$ by $D$.
An illustration of this can be found in Figure~\ref{fig:thm_fp_nni2a}.

\begin{figure}[H]
	\begin{subfigure}[b]{.45\textwidth}
		\centering
		\includegraphics[width=0.9\linewidth]{thm_fp_nni2a.eps}
		\vspace{12pt}
		\caption{path $p$ following $T$}
		\label{fig:thm_fp_nni2a}
	\end{subfigure}
	\begin{subfigure}[b]{.45\textwidth}
		\centering
		\includegraphics[width=0.9\linewidth]{thm_fp_nni2b.eps}
		\vspace{12pt}
		\caption{path $p'$ following $T'$}
		\label{fig:thm_fp_nni2b}
	\end{subfigure}
	\caption{Comparison of $p$ and $p'$ if there is an $\nni$ move on the edge $((T)_{t+2},T_{t+1})$ in $T$.
	At the bottom all possibilities for $T_2$ and $T'_2$, respectively, are displayed, depending on the cluster $C_k$ that is currently considered on $\findpath$:
	${C_k \subseteq A \cup D}$ on the left, ${C_k \subseteq B \cup D}$ in the middle, ${C_k \subseteq C \cup D}$ on the right.}
\end{figure}

We will now consider each $\nni$ move that could possibly happen on $\findpath$ on $T$ resulting in $T_1$, separately.
If $T_1$ is the tree at the top left of Figure~\ref{fig:thm_fp_nni2a}, resulting from $T$ by exchanging the subtrees induced by $C$ and $D$, the cluster $C_k$, the most recent common ancestor of which is moved down by $\findpath$, contains elements of $D$ and of $A \cup B$.
In the following we distinguish three cases, depending on whether $C_k$ contains, besides taxa of $D$, just taxa of $A$ or $B$, or both.
We denote the cluster $(R)_{k-1}$ by $C_{k-1}$.

\begin{enumerate}
    \item If $C_k$ contains elements of all three clusters $A, B$ and $D$, then $C_{k-1} = A \cup B$ as the two clusters of $R$ that join to $C_k$ in $R$ must already be present in $T$.
	\todo{LC: Is this clear?}
    As this $C_{k-1} = A \cup B$ is no cluster of $T'$, the move $\findpath$ makes on $T'$ decreases the rank of $(C_{k-1})_{T'}$ before $C_k$ is considered by the algorithm.
    This $\rnni$ move on $T'$ decreases the rank of $(C_{k-1})_{T'}$ by building the cluster $A \cup B$, such that $T'_1 = T$.
	This however contradicts $|\fp(T',R)| < |\fp(T,R)| - 1$.

    \item If $C_k \subseteq A \cup D$, the move on $T_1$ decreases the rank of $(C_k)_{T_1}$ by exchanging $B$ and $D$, resulting in the tree  $T_2$ as depicted in the bottom left of Figure~\ref{fig:thm_fp_nni2a}.
	As $C_k \subseteq A \cup D$, we can follow that from $T'$ to $T'_1$ to $T'_2$ the rank of $(C_k)_{T'}$ decreases by two by exchanging the subtree induced by $D$ with the ones induced by $B$ and $C$ as depicted on the left of Figure~\ref{fig:thm_fp_nni2b}.
	It follows that $T_2$ and $T'_2$ only differ by one interval (dotted edges in the trees at the bottom left of Figures~\ref{fig:thm_fp_nni2a} and~\ref{fig:thm_fp_nni2b}), and hence are $\rnni$ neighbours.
	This together with the fact that $|\fp(T_2,R)| = |\fp(T,R)|-2$ and  $|\fp(T'_2,R)| = |\fp(T',R)|-2$ contradicts the assumption that $\fp(T,R)$ has minimum length among all $\findpath$ paths between pairs of trees violating inequality~(\ref{eqn:iff_inequality}).

	\item
	The case $C_k \subseteq B \cup D$ is analogous to the previous one.
    The two $\rnni$ moves following $T$ and $T'$ on $p$ and $p'$, respectively, end up in trees $T_2$ and $T'_2$ as depicted in the trees in the middle at the bottom of Figures~\ref{fig:thm_fp_nni2a} and \ref{fig:thm_fp_nni2b} that are $\rnni$ neighbours.
    As in the previous case, $T_2, T'_2$ and $R$ violate inequality~(\ref{eqn:iff_inequality}) and $|\fp(T_2,R)| < |\fp(T,R)|$, which contradicts our assumptions on $T$ and $R$.
\end{enumerate}

The second $\nni$ move possible on $((T)_{t+2},(T)_{t+1})$ builds the cluster $C \cup D$ in $T_1$ as illustrated on the right of Figure~\ref{fig:thm_fp_nni2a}.
Hence it is $C_k \subseteq C \cup D$.
If $(C_k)_T$ does not move further down on $p$, $C_{k-1} = A \cup B$ is a cluster in $R$.
As this cluster is not present in $T'$, the $\rnni$ move that $\findpath$ does on this tree builds the cluster $A \cup B$, resulting in $T'_1 = T$, which contradicts $|\fp(T',R)| < |\fp(T,R)| - 1$.
If on the other side the rank of $(C_k)_{T_1}$ decreases further on $p$, the rank of $(C_k)_{T'}$ decreases by at least two on $p'$ as well.
As $C_k \subseteq C \cup D$, these two moves on $p'$ are $\nni$ moves exchanging the subtree induced by $D$ with the ones induced by $B$ and $A$.
These moves are shown on the right of Figure~\ref{fig:thm_fp_nni2b}.
We end up in the same contradiction as in the previous case as the trees $T_2, T'_2$ and $R$ violate inequality~(\ref{eqn:iff_inequality}) and $|\fp(T_2,R)| < |\fp(T,R)|$.

\item Rank move on interval $[(T)_{t+2},(T)_{t+1}]$

Notice that this is only possible if this is a rank interval.
If this rank move is followed by a further rank move, decreasing the rank of $(C_k)_{T_1}$ further, there are two rank moves decreasing the rank of $(C_k)_{T'}$ on $p'$ as well.
The trees $T_2$ and $T'_2$ after these two moves on $p$ and $p'$ coincide in all but one interval.
As previously, this contradicts the assumption that $T$ and $R$ are the closest trees that violate inequality~(\ref{eqn:iff_inequality}).
If on the other side there is no such rank move on $T_1$, it is $C_{k-1} = A \cup B$.
It follows $T'_1 = T$ as this tree results from building $C_{k-1}$ in $T'$, which happens right before $C_k$ is considered by $\findpath$ on $p'$.
This contradicts $|\fp(T',R)| < |\fp(T,R)| - 1$.
\todo{LC: Add a figure for this case?}

\item $\rnni$ move on interval below $[(T)_{t},(T)_{t-1}]$

In this case it is $C_k \subseteq A \cup B$.
Hence the move decreasing the rank of $(C_k)_{T'}$ is an $\nni$ move exchanging $B$ and $C$.
This results in $T'_1 = T$, which contradicts $|\fp(T',R)| < |\fp(T,R)| - 1$.
\end{enumerate}

Let us now assume that there is a rank move between $T$ and $T'$ where nodes $(T)_{t+1}$ and $(T)_t$ inducing clusters $A$ and $B$ swap ranks as depicted at the top of Figure~\ref{fig:thm_fp_rank1}.
We will again consider the different moves from $T$ to $T_1$ that are possible on intervals incident to $(T)_{t+1}$ or $(T)_{t}$ and show that each of these cases ends in a contradiction, proving that $T,T'$ and $R$ cannot violate inequality~(\ref{eqn:iff_inequality}).

\begin{figure}[!hbt]
\centering
\includegraphics[width=0.4\textwidth]{thm_fp_rank1}
\vspace{12pt}
\caption{Rank move between $T$ and $T'$ on the interval given by the highlighted nodes, and an $\nni$ move on $T$ if the edge $((T)_{t+2},(T)_{t+1})$ has length one.}
\label{fig:thm_fp_rank1}
\end{figure}

\begin{enumerate}
\item Rank move on $[(T)_{t+1},(T)_t]$

This move on $T$ results in $T_1 = T'$.
It follows $|\fp(T',R)| = |\fp(T,R)| - 1$, which contradicts $|\fp(T',R)| < |\fp(T,R)| - 1$.

\item $\nni$ move on edge above $[v,w]$

Notice that this is only possible if the interval above $[v,w]$ is an edge $(u,v)$ of length one.
We are now going to distinguish the case that $u$ is the parent of $w$ from the case that it is not.
At first we assume that $u$ is parent of $w$.
It follows, that after the $\nni$ move on $(u,v)$, there is a new cluster containing $A$ and $B_1$, which is one of the subtrees that are children of $v$.
This move is depicted on the left of Figure~\ref{fig:thm_fp_rank1}.
If such a move happens the currently considered cluster $C_k$ must be subset of $A \cup B_1$, which means that $C_{k-1} = A$ is already at the same place in $T$ as in $R$.
Therefore, the move on $p'$ on $T'$ must be a rank swap leading to $T$, as $(A)_{T'}$ is moved to its correct position before $C_k$ is considered, which follows from how $\findpath$ works.
However, this means that $T$ is on the path $p'$ form $T'$ to $R$, which is a contradiction to $|\fp(T',R)| < |\fp(T,R)| - 1$

Let us now assume that $u$ is not the parent of $w$.
In this case the $\nni$ move builds a new cluster containing taxa of $B$ and of the cluster $C$ that is induced by the child of $u$ that is not $v$, as depicted in the top left of Figure~\ref{fig:thm_fp_rank2}.
It follows that $C_k \subseteq C \cup B_1$ where $B_1$ is one of the clusters that is induced by a child of $w$.
If there was no rank move on $T_1$ decreasing the rank of $(C_k)_{T_1}$ further, it must be $C_{k-1} = A$ meaning that $A$ is in the same position in $T$ as in $R$.
In this case the tree following $T'$ on $p'$ is be $T$ as $C_{k-1} = A$ moves to it's final position on $\findpath$ before $C_k$ is moved down.
However, this contradicts $|\fp(T',R)| < |\fp(T,R)| - 1$.
Let us now consider what happens if there is a rank move on $T_1$ that decreases the rank of $(C_k)_{T_1}$ on $p$.
This case is depicted on the left of Figure~\ref{fig:thm_fp_rank2}.
The moves happening on $p'$ are as follows.
As it is $C_k \subseteq C \cup B$, the rank of $(C_k)_{T'}$ decreases by a rank swap of $(C \cup B)_{T'}$ and $(A)_{T'}$ on $T'$ first and then an $\nni$ move exchanges $B_2$ and $C$, as it is depicted on the right in the same figure.
One can easily see that the two trees on $p$ and $p'$ that are two $\rnni$ moves apart from $T$ and $T'$, respectively, only differ by one interval.
Again, this is a contradiction to the fact that $T$ and $R$ are the trees with minimum distance for which we can find a $T' \in N_1(T)$ with $|\fp(T',R)| < |\fp(T,R)| - 1$.

\begin{figure}[!hbt]
	\centering
	\includegraphics[width=0.4\textwidth]{thm_fp_rank2}
	\vspace{12pt}
	\caption{Comparison of $p$ (left) and $p'$ (right) if there is a rank move between $T$ and $T'$ and an $\nni$ move on the edge below the corresponding rank interval follows on $p$.}
	\label{fig:thm_fp_rank2}
\end{figure}

\item Rank move on interval above $[v,w]$

Notice that this is only possible if the interval above $[v,w]$ is a rank interval.
If there is a rank move increasing the rank of $v$, and no rank move increasing the rank of $w$ immediately afterwards, $C_k$ reaches it's final position in the tree $T_1$ on $p$ following $T$.
It follows that the cluster $A$ induced by $w$ in $T$ is the cluster $C_{k-1}$ in $R$.
This means that the move following $T'$ on $p'$ leads to $T$, as the node inducing $A$ moves down before $C_k$ is considered on $\findpath$.
This contradicts $|\fp(T',R)| < |\fp(T,R)| - 1$.
If on the other side the rank swap on $T$ is directly followed by a rank swap increasing the rank of $w$, such rank swaps happen on $p'$ as well and the trees following $T$ and $T'$ after two $\rnni$ moves on $p$ and $p'$, respectively, coincide in all but one interval.
This is again a contradiction to our choice of $T$ and $R$.

\item $\rnni$ moves on interval below $[v,w]$

If there is a move on the interval below $[v,w]$, it follows that $\findpath$ moves $C_k \subseteq A$ down where $A$ is the cluster induced by $w$.
For decreasing the rank of the most recent common ancestor of $C_k \subseteq A$ in $T'$, the move on $T'$ must be a rank swap that results in $T$, contradicting  $|\fp(T',R)| < |\fp(T,R)| - 1$.
\end{enumerate}

We can conclude that in any of the above cases we end up in a contradiction, proving that a tree $T'$ with $|\fp(T',R)| < |\fp(T,R)| - 1$ does not exist in $N_1(T)$.
Therefore we can follow that $|\fp(T',R)| \geq |\fp(T,R)| -1$ for all $T' \in N_1(T)$.
\endproof

\end{document}
